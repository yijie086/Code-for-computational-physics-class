%!TEX program = xelatex
\documentclass{article}
\usepackage[UTF8]{ctex}
\usepackage{authblk}
\usepackage{listings}
\usepackage{subfigure}
\usepackage{amsmath, amsfonts, amsthm, amssymb, graphicx, xcolor, hyperref, mathrsfs}
\lstset{
    basicstyle          =   \sffamily,          % 基本代码风格
    keywordstyle        =   \bfseries,          % 关键字风格
    commentstyle        =   \rmfamily\itshape,  % 注释的风格,斜体
    stringstyle         =   \ttfamily,  % 字符串风格
    flexiblecolumns,                % 别问为什么,加上这个
    numbers             =   left,   % 行号的位置在左边
    showspaces          =   false,  % 是否显示空格,显示了有点乱,所以不现实了
    numberstyle         =   \zihao{-5}\ttfamily,    % 行号的样式,小五号,tt等宽字体
    showstringspaces    =   false,
    captionpos          =   t,      % 这段代码的名字所呈现的位置,t指的是top上面
    frame               =   lrtb,   % 显示边框
}

\title{计算物理作业4} %文章标题
\author[a]{王一杰} %作者的名称
\affil[a]{中国科学技术大学}
\date{\today}%日期
\usepackage [a4paper, left=25mm,right=25mm,top=15mm,bottom=20mm]{geometry} %设置页面的环境,a4纸张大小,左右上下边距信息

\begin{document}
\maketitle
%添加这一句才能够显示标题等信息
%\tableofcontents

\section{Homework 4}
\subsection{Problem 1}
根据RK4算法,该问题的程序很容易给出如下(MATLAB代码):
\begin{lstlisting}[language=MATLAB]
x0=2;					%设定初始值
t1=3;					%设定计算时间
dt=0.2;					%计算步长
xstore=zeros(1,t1/dt+1);		%计算结果储存
xstore(1,1)=x0;			
t=0:dt:t1;				%计算时间节点储存
for i=1:1:(t1/dt)			%RK4算法
    K1=10-3*xstore(1,i);
    K2=10-3*(xstore(1,i)+dt*K1*0.5);
    K3=10-3*(xstore(1,i)+dt*K2*0.5);
    K4=10-3*(xstore(1,i)+dt*K3);
    xstore(1,i+1)=xstore(1,i)+dt*(K1+K2*2+K3*2+K4)/6;
end

\end{lstlisting}
演化结果如图1所示,计算结果与理论计算期望$x=\frac{10-4e^{-3t}}{3}$相符,且在$dt$较大($dt=0.2$)时已经取得相当好的结果。
\begin{figure}[tbp]
  \center{\includegraphics[width=8cm]{pro1.eps}}
 % \subfigure[Problem2结果]{\includegraphics[width=8cm]{pro2.eps}}
 \caption{Problem1的程序计算结果}
\end{figure}

\subsection{Problem 2}
给出递推公式$x_{i+1}=x_i+dt\cdot(\lambda_1 K_1+\lambda_2 K_2+ \lambda_3 K_3)$,

且取$K_1=f(t_i,x_i),\ K_2=f(t_i+p_2\cdot dt,x_i+p_2\cdot dtK_1),\ K_3=f(t_i+p_3\cdot dt,x_i+p_{31}\cdot dtK_1+p_{32}\cdot dtK_2)$

其中,$K_1=\dot{x}(t_{i+1}),$

$K_2= \dot{x} ( t _ { i } ) +p_2\cdot dt\frac{\partial f}{\partial t}+p_2\cdot dtK_1\frac{\partial f}{\partial x}+\frac{1}{2}(p_2\cdot dt)^2\frac{\partial^2 f}{\partial t^2}+\frac{1}{2}(p_2\cdot dt K_1)^2\frac{\partial^2 f}{\partial x^2}+(p_2\cdot dt)^2K_1 \frac{\partial^2f}{\partial t \partial x}+O(dt^3),$

$K_3=\dot{x} ( t _ { i } ) +p_2\cdot dt\frac{\partial f}{\partial t}+(p_{31}\cdot dtK_1+p_{32}\cdot dtK_2)\frac{\partial f}{\partial x}+\frac{1}{2}(p_2\cdot dt)^2\frac{\partial^2 f}{\partial t^2}+\frac{1}{2}(p_{31}\cdot dtK_1+p_{32}\cdot dtK_2)^2\frac{\partial^2 f}{\partial x^2}+(p_2\cdot dt)\cdot(p_{31}\cdot dtK_1+p_{32}\cdot dtK_2) \frac{\partial^2f}{\partial t \partial x}+O(dt^3),$

代入$x_{i+1}=x_i+dt\cdot(\lambda_1 K_1+\lambda_2 K_2+ \lambda_3 K_3)+O(dt^4)$,
与Taylor展开公式$x_{i+1}=x_i+dt\dot{x}(t_i)+\frac{dt^2}{2}\ddot{x}(t_i)+\frac{dt^3}{6}\dddot{x}(t_i)+O(dt^4)$对比,可得要求满足的方程如下:

$\lambda_1+\lambda_2+\lambda_3=1,$

$p_3=p_{31}+p_{32},$

$p_2\lambda_2+p_3\lambda_3=\frac{1}{2},$


$p_2^2\lambda_2+p_3^2\lambda_3=\frac{1}{3},$

$p_2p_{32}\lambda_3=\frac{1}{6}.$

上述方程显然有无穷多组解,能够满足$O(dt^4)$精度,命题得证。
\end{document}	
