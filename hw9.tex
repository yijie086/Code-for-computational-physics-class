%!TEX program = xelatex
\documentclass{article}
\usepackage[UTF8]{ctex}
\usepackage{authblk}
\usepackage{listings}
\usepackage{subfigure}
\usepackage{amsmath, amsfonts, amsthm, amssymb, graphicx, xcolor, hyperref, mathrsfs}
\lstset{
    basicstyle          =   \sffamily,          % 基本代码风格
    keywordstyle        =   \bfseries,          % 关键字风格
    commentstyle        =   \rmfamily\itshape,  % 注释的风格,斜体
    stringstyle         =   \ttfamily,  % 字符串风格
    flexiblecolumns,                % 别问为什么,加上这个
    numbers             =   left,   % 行号的位置在左边
    showspaces          =   false,  % 是否显示空格,显示了有点乱,所以不现实了
    numberstyle         =   \zihao{-5}\ttfamily,    % 行号的样式,小五号,tt等宽字体
    showstringspaces    =   false,
    captionpos          =   t,      % 这段代码的名字所呈现的位置,t指的是top上面
    frame               =   lrtb,   % 显示边框
    breaklines%自动换行
}

\title{计算物理作业9} %文章标题
\author[a]{王一杰} %作者的名称
\affil[a]{中国科学技术大学}
\date{\today}%日期
\usepackage [a4paper, left=25mm,right=25mm,top=20mm,bottom=20mm]{geometry} %设置页面的环境,a4纸张大小,左右上下边距信息

\begin{document}
\maketitle
%添加这一句才能够显示标题等信息
\tableofcontents

\section{Problem 1}
\subsection{程序设定}
用二维伊辛模型来描述铁磁相变问题,已知相变发生
在温度 $T=2.26$ 时,温度低于相变点为铁磁性,原子自旋排布相对有序,高于相变点为顺磁性,自旋排布无序。
本题考察 $40\times 40$ 的二维原子点阵,每个原子的自旋向上$(=1)$或向下$(=0)$。每个数据点为二维阵列自旋信息,对应标签为铁磁$(=1)$或顺磁$(=0)$。利用机器学习随机森林方法来解决这个分类问题。
已用蒙特卡罗方法模拟出数据样本,从 $T=0.25\rightarrow 4$ 取 $16$ 个温度点,每个温度下有 $10000$ 个数据样本。

解决如下问题:

- 将测试、训练样本上预言正确率随训练样本占总样
本比例的关系画出,训练样本比例区间为 $0.1\%-90\%$;

- 确定训练样本$90\%$,测试$10\%$,研究正确率随树木颗数、树木深度,以及叶子上最小样本数的关系,画图。最后给出最优超参选择。

\subsection{程序实现}

程序实现由参考网站即可。

代入参数,得到结果如图1所示:

\begin{figure}[t]
 \subfigure[预言正确率随训练样本占总样
本比例的关系]{\includegraphics[width=8cm]{hw91-ratioscore.eps}}
 \subfigure[预言正确率随树木颗数的关系]{\includegraphics[width=8cm]{hw91-noftree.eps}}
 \subfigure[预言正确率随树木深度的关系]{\includegraphics[width=8cm]{hw91-treedep.eps}}
 \subfigure[预言正确率随叶子上最小样本数的关系]{\includegraphics[width=8cm]{hw91-sampleperleaf.eps}}
 \caption{Problem 1 运行结果}
\end{figure}

\subsection{结果分析}

(1)从图1(a)中可以看出,预言正确率随训练样本占总样本比例的关系在初期的提升比较明显,后期不明显,且有一定的震荡。

(2)从图1(b)(c)(d)中可以分析一个可能较优的超参为:树木颗数$\approx 10^{1.2}$,树木深度$\approx 10^{0.5}$,叶子上最小样本数$\approx 10^{0}$


\section{Problem 2}
\subsection{程序设定}
寻找新物理现象是基础物理前沿方向。考察在对撞机实验中寻找超对称现象。利用蒙特卡洛方法产生了共五百万个数据样本,包含信号和噪声(即标签)。每个数据点对应实验末态的 $18$ 特征。利用深度神经元网络来研究此分类问题。模拟数据样本可从此处下载。建议使用 Pytorch 程序包,相似例子可见参考书网页。
解决如下问题:
   
- 使用单层隐藏神经元 $1000$ 个,研究预言正确率与 训练样本大小的关系,训练样本数目范围 $1000 \rightarrow 4500000$,画出关系图;

- 固定隐藏层神经元每层 $100$ 个,研究正确率与隐藏层数的关系,层数范围 $1-5$,画出关系图。
\subsection{程序实现}

程序实现由参考网站即可。

代入参数,得到结果如图2所示:

\begin{figure}[t]
 \subfigure[预言正确率随训练样本数目的关系]{\includegraphics[width=8cm]{hw92-datasize.eps}}
 \subfigure[预言正确率随隐藏神经元层数的关系]{\includegraphics[width=8cm]{hw92-numlayers.eps}}
 \caption{Problem 2 运行结果}
\end{figure}

\subsection{结果分析}

(1)从图2(a)中可以看出随着训练样本数目的增加,预言正确率前期上升很快,后期基本不变,维持在$80\%$左右。

(2)从图2(b)中可以看出,隐藏神经元层数在这个问题中和该参数环境下,与正确率基本无关,维持在$80\%$左右。

\end{document}	